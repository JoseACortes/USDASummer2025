\begin{abstract}
Soil carbon is a key component of soil health and plays a crucial role in the global carbon cycle. 
Accurate measurement of carbon is essential for measuring soil quality and its impact on the environment. 
Traditional methods for measuring carbon are often time-consuming, expensive, and require laboratory analysis. 
In situ Neutron-Gamma spectral analysis (NGSA) offers an alternative for rapid and non-destructive measurement of soil carbon. 
This paper explores the use of NGSA techniques to measure carbon levels in various soil types. 
We simulate in MCNP6.2 common soil types and apply different NGSA methods, including peak fitting, component fitting, singular value decomposition, and deep learning, to evaluate their effectiveness in measuring of soil carbon. 
The results demonstrate that peak fitting with exponential falloff baseline achieves the lowest mean squared error ($7.66\times10^{-5}$), followed by component analysis methods. 
The study shows that NGSA methods can provide accurate soil carbon measurements, with convolution techniques improving overall accuracy across all methods.
\end{abstract}

\begin{keyword}
soil carbon \sep spectral analysis 
\sep gamma-ray spectroscopy \sep MCNP simulation 
\sep peak fitting \sep component analysis \sep deep learning
\end{keyword}