\subsection{Common Soil Types}

To investigate the effectiveness of NGSA methods for soil carbon measurement, we simulate a range of common soil types. The simulated data includes spectral readings across different wavelengths, capturing the unique spectral signatures of each soil type. This data serves as a foundation for applying various NGSA techniques.

\begin{table}[H]
\centering
\caption{Material compositions for soil simulation}
\label{tab:materials}
\begin{tabular}{lrrrrrrr}
\hline
 Material   &   C \% &   H \% &   O \% &   Si \% &   Na \% &   Al \% &   K \% \\
\hline
 Carbon     & 100.0 &   0.0 &   0.0 &    0.0 &    0.0 &    0.0 &   0.0 \\
 Water      &   0.0 &  11.2 &  88.8 &    0.0 &    0.0 &    0.0 &   0.0 \\
 Quartz     &   0.0 &   0.0 &  53.3 &   46.7 &    0.0 &    0.0 &   0.0 \\
 Feldspar   &   0.0 &   0.0 &  48.8 &   32.1 &    8.8 &   10.3 &   0.0 \\
 Mica       &   0.0 &   0.5 &  48.2 &   21.2 &    0.0 &   20.3 &   9.8 \\
\hline
\end{tabular}
\end{table}

Table \ref{tab:materials} presents the elemental composition of common soil materials used in the simulations. Mechanical mixing is used to combine materials based on their weight proportions, such that a 50\% carbon and 50\% water mix would have a composition of 50\% carbon, 5.6\% hydrogen, and 44.4\% oxygen. To measure the effectiveness of NGSA methods for carbon measurement, we simulate combinations of soil materials with varying carbon (C) and moisture (Water) content.