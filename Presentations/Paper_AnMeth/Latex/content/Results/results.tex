\section{Results}

The effectiveness of each method in measuring carbon levels is evaluated based on accuracy using mean squared error (MSE) as the metric. The results are summarized in Table~\ref{tab:results}.

\begin{table}[H]
\centering
\caption{Performance comparison of spectral analysis methods}
\label{tab:results}
\begin{tabular}{llc}
\toprule
Method Group & Method & MSE \\
\midrule
Peak Fitting & Exponential Falloff Baseline & $7.66 \times 10^{-5}$ \\
Component Analysis & Elemental Maps & $2.10 \times 10^{-4}$ \\
Component Analysis & Average Training & $3.43 \times 10^{-4}$ \\
Peak Fitting & Linear Baseline & $3.52 \times 10^{-4}$ \\
Machine Learning & Deep Learning & $3.67 \times 10^{-4}$ \\
Convex Optimization & SVD & $4.27 \times 10^{-4}$ \\
\bottomrule
\end{tabular}
\end{table}

\subsection{Comparing Analysis Methods}

Peak fitting with exponential falloff baseline is the most effective method for measuring carbon levels in soil, achieving the lowest MSE ($7.66 \times 10^{-5}$). Component analysis methods also perform well, with elemental maps achieving the second-best performance. The deep learning method shows promise but requires further optimization to improve its performance.

\subsection{Effects of Carbon Levels on Results}

\begin{table}[H]
\centering
\caption{Method performance by carbon level}
\label{tab:carbon_level_effects}
\begin{tabular}{lccccccc}
\toprule
Carbon Level & \multicolumn{7}{c}{MSE Values} \\
\cmidrule(lr){2-8}
& Peak Fitting & Peak Fitting & Component & Component & Convex & Filtered ML & Machine \\
& (Exp Falloff) & (Linear) & (Average) & (Elemental) & Optimization & & Learning \\
\midrule
Agricultural & $7.66 \times 10^{-5}$ & $3.52 \times 10^{-4}$ & $3.43 \times 10^{-4}$ & $2.10 \times 10^{-4}$ & $4.27 \times 10^{-4}$ & $3.70 \times 10^{-3}$ & $3.67 \times 10^{-4}$ \\
All & $1.42 \times 10^{-2}$ & $3.48 \times 10^{-2}$ & $1.92 \times 10^{-2}$ & $1.92 \times 10^{-2}$ & $2.64 \times 10^{-2}$ & $1.33 \times 10^{-1}$ & $1.01 \times 10^{-1}$ \\
\bottomrule
\end{tabular}
\end{table}

Lower carbon levels tend to result in higher MSE values across all methods, indicating that the spectral signatures of low-carbon soils are less distinct and more challenging to analyze accurately. The methods generally perform better with higher carbon concentrations, where the spectral features are more pronounced.

\subsection{Effects of Convolution on Results}

\begin{table}[H]
\centering
\caption{Method performance by training dataset type}
\label{tab:convolution_effects}
\begin{tabular}{lccccccc}
\toprule
Dataset & \multicolumn{7}{c}{MSE Values} \\
\cmidrule(lr){2-8}
& Peak Fitting & Peak Fitting & Component & Component & Convex & Filtered ML & Machine \\
& (Exp Falloff) & (Linear) & (Average) & (Elemental) & Optimization & & Learning \\
\midrule
Convolution Training & $7.26 \times 10^{-5}$ & $3.50 \times 10^{-4}$ & $2.95 \times 10^{-4}$ & $1.91 \times 10^{-4}$ & $2.93 \times 10^{-4}$ & $3.65 \times 10^{-4}$ & $3.61 \times 10^{-4}$ \\
Feldspar & $3.43 \times 10^{-5}$ & $3.60 \times 10^{-4}$ & $7.48 \times 10^{-7}$ & $7.92 \times 10^{-7}$ & $2.14 \times 10^{-3}$ & $1.10 \times 10^{-2}$ & $9.30 \times 10^{-4}$ \\
Material Mixes & $7.66 \times 10^{-5}$ & $3.52 \times 10^{-4}$ & $3.43 \times 10^{-4}$ & $2.10 \times 10^{-4}$ & $4.27 \times 10^{-4}$ & $3.70 \times 10^{-3}$ & $3.67 \times 10^{-4}$ \\
\bottomrule
\end{tabular}
\end{table}

Convolution generally improves the accuracy of spectral analysis methods by smoothing out noise and enhancing the signal-to-noise ratio. The results show that convolution leads to lower MSE values across all methods, indicating that it is beneficial for spectral analysis in soil carbon measurement.