\documentclass[review]{elsarticle}

\usepackage{lineno,hyperref}
\usepackage{amsmath}
\usepackage{amssymb}
\usepackage{graphicx}
\usepackage{booktabs}
\usepackage{multirow}
\usepackage{array}
\usepackage{float}
\usepackage[utf8]{inputenc}

% For line numbers
\modulolinenumbers[5]

\journal{Radiation Physics and Chemistry}

%%%%%%%%%%%%%%%%%%%%%%%
%% Elsevier bibliography style
%%%%%%%%%%%%%%%%%%%%%%%
%% To change the style, put a % in front of the second line of the current style and
%% remove the % from the first line of the style you would like to use.

%% Numbered
%\bibliographystyle{model1-num-names}

%% Numbered without titles
%\bibliographystyle{model1a-num-names}

% % Harvard
% \bibliographystyle{model2-names.bst}\biboptions{authoryear}

%% Vancouver numbered
%\usepackage{numcompress}\bibliographystyle{model3-num-names}

%% Vancouver name/year
%\usepackage{numcompress}\bibliographystyle{model4-names}\biboptions{authoryear}

%% APA style
%\bibliographystyle{model5-names}\biboptions{authoryear}

%% AMA style
%\bibliographystyle{model6-num-names}

%% `Elsevier LaTeX' style
\bibliographystyle{elsarticle-num}
%%%%%%%%%%%%%%%%%%%%%%%

\begin{document}

\begin{frontmatter}

%% Title, authors and addresses

%% use the tnoteref command within \title for footnotes;
%% use the tnotetext command for theassociated footnote;
%% use the fnref command within \author or \address for footnotes;
%% use the fntext command for the associated footnote;
%% use the corref command within \author for corresponding author footnotes;
%% use the cortext command for the associated footnote;

\title{In Situ Spectral Analysis for Soil Carbon Measurement}

%% use optional labels to link authors explicitly to addresses:
%% \author[label1,label2]{}
%% \address[label1]{}
%% \address[label2]{}

\author[uta]{Jose A Cortes\corref{cor1}}
\ead{jose.cortes@uta.edu}

\author[uta]{Andrzej Korzeniowski}
\ead{andrzej.korzeniowski@uta.edu}

\author[usda]{Galina Yakubova}
\ead{galina.yakubova@usda.gov}

\author[usda]{Aleksandr Kavetskiy}
\ead{aleksandr.kavetskiy@usda.gov}

\author[usda]{Allen Tobert}
\ead{allen.tobert@usda.gov}

\cortext[cor1]{Corresponding author}

\address[uta]{University of Texas at Arlington, Arlington TX, USA}
\address[usda]{USDA ARS Auburn Lab, Auburn AL, USA}

\begin{abstract}
Soil carbon is a key component of soil health and plays a crucial role in the global carbon cycle. Accurate measurement of carbon is essential for measuring soil quality, and its impact on the environment. Traditional methods for measuring carbon are often time-consuming, expensive, and require laboratory analysis. In situ spectral analysis offers an alternative for rapid and non-destructive measurement of soil carbon. This paper explores the use of spectral analysis techniques to measure carbon levels in various soil types. We simulate common soil types and apply different spectral analysis methods, including peak fitting, component fitting, singular value decomposition, and deep learning, to evaluate their effectiveness in measuring soil carbon. The results demonstrate the potential of spectral analysis methods for measuring soil carbon levels, with peak fitting and component fitting methods showing the best performance.
\end{abstract>}

\begin{keyword}
soil carbon \sep spectral analysis \sep in situ measurement \sep MCNP simulation \sep peak fitting \sep component fitting \sep singular value decomposition \sep deep learning
\end{keyword}

\end{frontmatter}

\linenumbers

\section{Introduction}

Soil carbon is a key component of soil health and plays a crucial role in the global carbon cycle. Accurate measurement of carbon is essential for measuring soil quality, and its impact on the environment~\cite{lal_soil_2018}. Traditional methods for measuring carbon are often time-consuming, expensive, and require laboratory analysis~\cite{smith_how_2020}. In situ spectral analysis offers an alternative for rapid and non-destructive measurement of soil carbon~\cite{yakubova_tagged_2019}. This paper explores the use of spectral analysis techniques to measure carbon levels in various soil types. We simulate common soil types and apply different spectral analysis methods, including peak fitting, component fitting, singular value decomposition, and deep learning, to evaluate their effectiveness in measuring soil carbon.

\section{Data Generation}

\subsection{Common Soil Types}

To investigate the effectiveness of spectral analysis methods for soil carbon measurement, we simulate a range of common soil types. The simulated data includes spectral readings across different wavelengths, capturing the unique spectral signatures of each soil type. This data serves as a foundation for applying various spectral analysis techniques.

The material compositions used in our simulations are detailed in Tables~\ref{tab:elements}, \ref{tab:compounds}, and \ref{tab:materials}.

\begin{table}[H]
\centering
\caption{Elemental properties used in MCNP simulations}
\label{tab:elements}
\begin{tabular}{@{}lcc@{}}
\toprule
Element & MCNP Identifier & Density (g/cm³) \\
\midrule
Si & 14028 & 2.33 \\
Al & 13027 & 2.7 \\
H & 1001 & 0.001 \\
Na & 11023 & 0.97 \\
O & 8016 & 0.00143 \\
Fe & 26000 & 7.87 \\
Mg & 12024 & 1.74 \\
C & 6000 & 2.33 \\
\bottomrule
\end{tabular}
\end{table}

\begin{table}[H]
\centering
\caption{Compound properties used in MCNP simulations}
\label{tab:compounds}
\begin{tabular}{@{}lc@{}}
\toprule
Compound & Density (g/cm³) \\
\midrule
SiO$_2$ & 2.65 \\
Al$_2$O$_3$ & 3.95 \\
H$_2$O & 1.0 \\
Na$_2$O & 2.16 \\
Fe$_2$O$_3$ & 5.24 \\
MgO & 2.74 \\
C & 2.33 \\
\bottomrule
\end{tabular}
\end{table}

\begin{table}[H]
\centering
\caption{Material compositions and densities}
\label{tab:materials}
\begin{tabular}{@{}lcc@{}}
\toprule
Material & Compound Makeup (by weight) & Density (g/cm³) \\
\midrule
Silica & SiO$_2$ (76.4\%), Al$_2$O$_3$ (23.6\%) & 2.32 \\
Kaolinite & SiO$_2$ (46.5\%), Al$_2$O$_3$ (39.5\%), H$_2$O (14.0\%) & 3.95 \\
Smectite & SiO$_2$ (66.7\%), Al$_2$O$_3$ (28.3\%), H$_2$O (5.0\%) & 2.785 \\
Montmorillonite & SiO$_2$ (73.7\%), Al$_2$O$_3$ (24.6\%), H$_2$O (1.7\%) & 2.7 \\
Quartz & SiO$_2$ (100.0\%) & 2.62 \\
Chlorite & SiO$_2$ (30.0\%), Al$_2$O$_3$ (24.0\%), Fe$_2$O$_3$ (23.3\%), H$_2$O (22.7\%) & 2.6 \\
Mica & SiO$_2$ (48.9\%), Al$_2$O$_3$ (40.3\%), H$_2$O (10.8\%) & 2.7 \\
Feldspar & SiO$_2$ (68.0\%), Al$_2$O$_3$ (32.0\%) & 2.55 \\
Coconut & C (100.0\%) & 0.53 \\
\bottomrule
\end{tabular}
\end{table}

To measure the effectiveness of spectral analysis methods for carbon measurement, we simulate combinations of soil materials with varying carbon content (coconut).

\subsection{Simulation in MCNP}

MCNP6~\cite{werner_mcnp_2017} was used to simulate gamma-ray spectra resulting from neutron activation of soil samples. Each simulation modeled a soil matrix with varying concentrations of carbon and other common soil constituents. The geometry was set up to mimic in situ measurement conditions, with a neutron source placed above a soil slab and a detector positioned to capture emitted gamma rays~\cite{kavetskiy_neutron_2017}.

\begin{figure}[H]
\centering
\includegraphics[width=0.8\textwidth]{../Figures/DataGeneration/MCNPGeometry.png}
\caption{Geometry of MCNP simulation setup}
\label{fig:mcnp_geometry}
\end{figure}

Key simulation parameters included:

\begin{itemize}
\item \textbf{Neutron source energy:} API120 portable neutron (D-T generator) generator~\cite{kavetskiy2018}
\item \textbf{Soil slab dimensions:} 112 cm × 90 cm × 30 cm
\item \textbf{Detector type:} Geiger-Mueller (G-M) detector~\cite{yakubova_measuring_2025}
\item \textbf{Tally:} F8 (pulse height tally) for gamma spectra
\end{itemize}

This approach enables the generation of realistic spectral data for a variety of soil compositions, forming the basis for evaluating different spectral analysis techniques.

\subsection{Spectral Readings}

The spectral readings obtained from the MCNP simulations provide a detailed representation of the gamma-ray emissions from the soil samples. Mathematically it is a probability density function (PDF) of the energy distribution of the emitted gamma rays.

\begin{figure}[H]
\centering
\includegraphics[width=0.8\textwidth]{../Figures/DataGeneration/MCNPSpectralReading.png}
\caption{MCNP spectral reading example}
\label{fig:mcnp_spectral}
\end{figure}

\subsection{Training and Testing Data}

The training data for the spectral analysis methods is picked from the edge cases of the simulated data. This includes the highest and lowest carbon levels both as would be found in simulation as well as natural soils. The testing data is all cases of the simulated data, excluding the training data.

\begin{table}[H]
\centering
\caption{Carbon level classifications}
\label{tab:carbon_levels}
\begin{tabular}{@{}lc@{}}
\toprule
Carbon Level & Associated Amount \\
\midrule
Natural & 0\%--6\% Carbon \\
High & 6\%--100\% Carbon \\
\bottomrule
\end{tabular}
\end{table}

\begin{figure}[H]
\centering
\includegraphics[width=0.8\textwidth]{../Figures/DataGeneration/FeldsparSpectralReadingByCarbonLevel.png}
\caption{Feldspar spectral reading by carbon level}
\label{fig:feldspar_carbon}
\end{figure}

\subsection{Data Convolution}

\begin{figure}[H]
\centering
\includegraphics[width=0.8\textwidth]{../Figures/DataGeneration/Sim_vs_Convoluted_FeldsparSpectralReadings_Combined.png}
\caption{Simulated vs convoluted data comparison}
\label{fig:sim_vs_conv}
\end{figure}

In the context of spectral analysis, MCNP can be used to simulate the interaction of radiation with soil materials, providing spectrums to analyze. Linear Convolution is used to quickly predict spectral readings for material mixtures by combining the spectral signatures of individual components. This does not account for the complex interactions between materials, but it provides a simplified approach to generate spectral data for analysis. The error metric for this convolution method is based on the difference between the simulated spectral readings and the readings obtained from MCNP simulations. The effects of convolution on the analysis results will be investigated in the results section.

\section{Analysis Methods of Spectral Readings}

This section explores various spectral analysis methods applied to the simulated spectral readings. Each method is evaluated for its effectiveness in measuring carbon levels. Error is calculated using mean squared error (MSE) between the predicted and actual carbon levels in the test data.

\subsection{Peak Fitting}

Peak fitting involves using the least-squares method in identifying and quantifying the baseline and peaks in the spectral data that correspond to specific soil components~\cite{gardner_use_2011}. This method is useful for extracting information about the concentration of individual elements or compounds in the soil. For effective peak fitting, the data is filtered to focus on the peak area.

\begin{table}[H]
\centering
\caption{Peak fitting function components}
\label{tab:peak_functions}
\begin{tabular}{@{}lll@{}}
\toprule
Symbol & Description & Example Function \\
\midrule
$F_p$ & Peak Function & Gaussian \\
$F_b$ & Baseline Function & Linear, Exp Falloff \\
$F_f$ & Fitting Function & $F_p + F_b$ \\
\bottomrule
\end{tabular}
\end{table}

\begin{table}[H]
\centering
\caption{Function parameterizations}
\label{tab:function_params}
\begin{tabular}{@{}lll@{}}
\toprule
Function Type & Parameterization & Example Expression \\
\midrule
Linear & Slope, Intercept & $ax + b$ \\
Exp Falloff & Amplitude, Decay, Offset & $a \cdot \exp(-b \cdot x) + c$ \\
Gaussian & Amplitude, Center, Width, Height & $a \cdot \exp(-((x - b)^{-p})^2) + c$ \\
\bottomrule
\end{tabular}
\end{table}

\begin{table}[H]
\centering
\caption{Peak fitting parameter bounds and starting values}
\label{tab:peak_param_bounds}
\begin{tabular}{@{}lllll@{}}
\toprule
Function Type & Parameter & Starting Parameter (p0) & Lower Bound & Upper Bound \\
\midrule
Linear & Slope & 0 & $-\infty$ & 0 \\
 & Intercept & window minimum & 0 & $\infty$ \\
Gaussian & Amplitude & gaus\_fix\_term $\times$ (window max - min) & 0 & (window max - min) \\
 & Center & Average of Bins & Min Bin & Max Bin \\
 & Width & (Max Bin - Min Bin)/6 & (Max Bin - Min Bin)/100 & (Max Bin - Min Bin)/2 \\
Exp Falloff & Center & Min Bin & $-\infty$ & $\infty$ \\
 & Amplitude & gaus\_fix\_term $\times$ (window max - min) & $-\infty$ & $\infty$ \\
 & Width & 1 & $-\infty$ & $\infty$ \\
 & Height & window minimum & $-\infty$ & $\infty$ \\
\bottomrule
\end{tabular}
\end{table}

This method relies on parameterized functions, which are fitted to the spectral data to identify the peaks corresponding to specific elements or compounds. The fitting function is a combination of a peak function (e.g., Gaussian) and a baseline function (e.g., linear or exponential falloff). Starting parameters are generated automatically such that the initial fitting function is within the bounds of the spectrum in the strong window. Parameters are also constrained to ensure they remain within reasonable limits based on the expected spectral characteristics of the soil components.

The Scipy python package is used for the fitting process~\cite{virtanen_scipy_2020}, leveraging its curve fitting capabilities to refine the initial parameter estimates. The baseline function is subtracted from the fitted function to isolate the peak, and the area under the peak is calculated to quantify the concentration of the corresponding element or compound in the soil.

\begin{figure}[H]
\centering
\includegraphics[width=0.8\textwidth]{../Figures/Analysis/peak_fitting_feldspar.png}
\caption{Fitted peak example for Feldspar sample}
\label{fig:peak_fitting}
\end{figure}

An activation layer of linear regression is used to compare the peak areas to known soil carbon concentrations, allowing for the calibration of the model's predictions.

\begin{figure}[H]
\centering
\includegraphics[width=0.8\textwidth]{../Figures/Analysis/PF_Exponential_Falloff_Agricultural_Carbon_Levels.jpg}
\caption{Peak Fitting prediction results for agricultural carbon levels}
\label{fig:peak_fitting_results}
\end{figure}

\subsection{Component Fitting}

Component fitting involves modeling the spectral data as a combination of known spectral signatures of soil components. This method allows for the estimation of the concentration of multiple components in the soil based on their spectral contributions.

Function: 
\begin{equation}
F_c = \sum_i A_i \cdot F_i
\end{equation}

Where:
\begin{itemize}
\item $F_c$ is the combined spectral function
\item $A_i$ are the coefficients representing the concentration of each component
\item $F_i$ are the spectral functions of individual components
\end{itemize}

Components can be any known spectral signature, this can be from pure elemental samples~\cite{kavetskiy_neutron_2023} or from the average of a set of soil samples. The fitting process involves adjusting the coefficients $A_i$ to minimize the difference between the combined spectral function $F_c$ and the observed spectral data. This method also benefits from filtering of low energy signals which are generally more likely to be caused by noise.

\begin{figure}[H]
\centering
\includegraphics[width=0.8\textwidth]{../Figures/Analysis/linear_combination_feldspar.png}
\caption{Component Fitting process applied to Feldspar sample}
\label{fig:component_fitting}
\end{figure}

The carbon coefficient $A_C$ is then used to estimate the carbon level in the soil. This method is particularly useful for analyzing complex soil mixtures where multiple known components contribute to the spectral signature. This method is also generalizable to study other elements or compounds.

\begin{figure}[H]
\centering
\includegraphics[width=0.8\textwidth]{../Figures/Analysis/CA_Agricultural_Carbon_Levels.jpg}
\caption{Component Analysis prediction results for agricultural carbon levels}
\label{fig:component_analysis_results}
\end{figure}

\subsection{Convex Optimization - Singular Value Decomposition (SVD)}

Convex Optimization is a mathematical technique used to decompose the spectral data into convex components~\cite{liu_deconvolution_2020}. The resulting singular values inside the strong window can be summed to provide a measure of the concentration of carbon in the soil.

\begin{figure}[H]
\centering
\includegraphics[width=0.8\textwidth]{../Figures/Analysis/decomposed_feldspar_svd.jpg}
\caption{SVD process applied to Feldspar sample}
\label{fig:svd_decomposition}
\end{figure}

\subsection{Deep Learning}

Deep learning techniques, such as convolutional neural networks (CNNs), can be applied to spectral data for feature extraction and classification. These methods can learn complex relationships in the data and provide robust predictions of carbon levels based on spectral readings. The most important difference between deep learning and the previous methods is that it requires a large amount of training data to be effective. One method by Kim et al.~\cite{kim_deep_2025} uses a deep learning model to predict existence, concentration and carbon peak areas.

\section{Results}

The effectiveness of each method in measuring carbon levels is evaluated based on accuracy using mean squared error (MSE) as the metric. The results are summarized in Table~\ref{tab:results_summary}.

\begin{table}[H]
\centering
\caption{Results summary for best performing methods}
\label{tab:results_summary}
\begin{tabular}{@{}llll@{}}
\toprule
method & carbon level & datasets used & mse \\
\midrule
Component Analysis - Average Training & Agricultural & Feldspar & 7.48497e-07 \\
Component Analysis - Elemental Maps & Agricultural & Feldspar & 7.91956e-07 \\
Peak Fitting - Exponential Falloff Baseline & Agricultural & Feldspar & 3.43383e-05 \\
Peak Fitting - Exponential Falloff Baseline & Agricultural & Convolution Training & 7.25777e-05 \\
Peak Fitting - Exponential Falloff Baseline & Agricultural & Material Mixes & 7.65971e-05 \\
Component Analysis - Elemental Maps & Agricultural & Convolution Training & 0.000191402 \\
Component Analysis - Elemental Maps & Agricultural & Material Mixes & 0.000209907 \\
SVD & Agricultural & Convolution Training & 0.000293279 \\
\bottomrule
\end{tabular}
\end{table}

% Peak Fitting Results Table
\begin{table}[H]
\centering
\caption{Peak Fitting results by method, carbon level, and dataset}
\label{tab:peak_fitting_results}
\begin{tabular}{@{}lllll@{}}
\toprule
method & carbon level & datasets used & mse \\
\midrule
Peak Fitting - Exponential Falloff Baseline & Agricultural & Feldspar & 3.43383e-05 \\
Peak Fitting - Exponential Falloff Baseline & Agricultural & Convolution Training & 7.25777e-05 \\
Peak Fitting - Exponential Falloff Baseline & Agricultural & Material Mixes & 7.65971e-05 \\
Peak Fitting - linear Baseline & Agricultural & Convolution Training & 0.000350061 \\
Peak Fitting - linear Baseline & Agricultural & Material Mixes & 0.000351535 \\
Peak Fitting - linear Baseline & Agricultural & Feldspar & 0.00036 \\
Peak Fitting - Exponential Falloff Baseline & All & Feldspar & 0.00194593 \\
Peak Fitting - Exponential Falloff Baseline & All & Convolution Training & 0.00331505 \\
Peak Fitting - linear Baseline & All & Convolution Training & 0.00585135 \\
Peak Fitting - Exponential Falloff Baseline & All & Material Mixes & 0.014231 \\
Peak Fitting - linear Baseline & All & Feldspar & 0.0187823 \\
Peak Fitting - linear Baseline & All & Material Mixes & 0.0347815 \\
\bottomrule
\end{tabular}
\end{table}

% Component Fitting Results Table
\begin{table}[H]
\centering
\caption{Component Fitting results by method, carbon level, and dataset}
\label{tab:component_fitting_results}
\begin{tabular}{@{}lllll@{}}
\toprule
method & carbon level & datasets used & mse \\
\midrule
Component Analysis - Average Training & Agricultural & Feldspar & 7.48497e-07 \\
Component Analysis - Elemental Maps & Agricultural & Feldspar & 7.91956e-07 \\
Component Analysis - Elemental Maps & Agricultural & Convolution Training & 0.000191402 \\
Component Analysis - Elemental Maps & Agricultural & Material Mixes & 0.000209907 \\
Component Analysis - Average Training & Agricultural & Convolution Training & 0.000295299 \\
Component Analysis - Average Training & Agricultural & Material Mixes & 0.000343431 \\
Component Analysis - Elemental Maps & All & Feldspar & 0.00302435 \\
Component Analysis - Average Training & All & Feldspar & 0.00303636 \\
Component Analysis - Elemental Maps & All & Convolution Training & 0.0037997 \\
Component Analysis - Average Training & All & Convolution Training & 0.00385569 \\
Component Analysis - Average Training & All & Material Mixes & 0.0191523 \\
Component Analysis - Elemental Maps & All & Material Mixes & 0.0192477 \\
\bottomrule
\end{tabular}
\end{table}

% SVD Results Table
\begin{table}[H]
\centering
\caption{SVD results by carbon level and dataset}
\label{tab:svd_results}
\begin{tabular}{@{}lll@{}}
\toprule
carbon level & datasets used & mse \\
\midrule
Agricultural & Convolution Training & 0.000293279 \\
Agricultural & Material Mixes & 0.000427226 \\
Agricultural & Feldspar & 0.0021363 \\
All & Convolution Training & 0.0106654 \\
All & Feldspar & 0.0115547 \\
All & Material Mixes & 0.0264083 \\
\bottomrule
\end{tabular}
\end{table}

% Deep Learning Results Table
\begin{table}[H]
\centering
\caption{Deep Learning results by carbon level and dataset}
\label{tab:deep_learning_results}
\begin{tabular}{@{}lll@{}}
\toprule
carbon level & datasets used & mse \\
\midrule
Agricultural & Material Mixes & 0.000358874 \\
Agricultural & Convolution Training & 0.0003602 \\
Agricultural & Feldspar & 0.000890089 \\
All & Convolution Training & 0.0889994 \\
All & Material Mixes & 0.101728 \\
All & Feldspar & 0.524807 \\
\bottomrule
\end{tabular}
\end{table}

\subsection{Comparing Analysis Methods}

The Peak Fitting method with Exponential Falloff Baseline is the most effective for measuring carbon levels in soil, achieving the lowest MSE. The Component Analysis methods also perform well, but are slightly less accurate than the best Peak Fitting approach. The Deep Learning method shows promise but requires further optimization to improve its performance.

\subsection{Effects of Carbon Levels on Results}

\begin{table}[H]
\centering
\caption{MSE values by carbon level and analysis method}
\label{tab:carbon_level_results}
\begin{tabular}{@{}lcccc@{}}
\toprule
Carbon Level & Peak Fitting - Linear & Peak Fitting - Exp Falloff & Component Analysis & SVD \\
\midrule
Agricultural & 0.000351535 & 7.65971e-05 & 0.000209907 & 0.000427226 \\
All & 0.0347815 & 0.014231 & 0.0191523 & 0.0264083 \\
\bottomrule
\end{tabular}
\end{table}

Lower carbon levels (agricultural range) tend to result in lower MSE values across all methods, indicating that the spectral signatures in the agricultural carbon range (0-6\%) are more distinct and easier to analyze accurately. The methods generally perform worse with higher carbon concentrations included in the analysis.

\subsection{Effects of Convolution on Results}

\begin{table}[H]
\centering
\caption{MSE values by dataset type}
\label{tab:convolution_results}
\begin{tabular}{@{}lcccc@{}}
\toprule
Datasets Used & Peak Fitting & Component Analysis & SVD & Machine Learning \\
\midrule
Material Mixes & 0.014231 & 0.0191523 & 0.0264083 & 0.0971477 \\
Convolution Training & 0.00331505 & 0.00385569 & 0.0106654 & 0.0893365 \\
Feldspar & 0.00194593 & 0.00303636 & 0.0115547 & 0.416227 \\
\bottomrule
\end{tabular}
\end{table}

Using the Feldspar dataset generally improves the accuracy of spectral analysis methods for peak fitting, while the Convolution Training dataset works better for component analysis and SVD. The Machine Learning method performs better with the Convolution Training dataset but shows poor performance with the Feldspar dataset, indicating that it requires more diverse training data to be effective.

\section{Discussion}

\subsection{Conclusions}

The study demonstrates the potential of spectral analysis methods for measuring soil carbon levels. Component fitting and peak fitting methods show the best performance, while deep learning techniques require further refinement. Convolution is beneficial for improving the accuracy of spectral analysis.

\subsection{Future Work}

Future work will focus on optimizing the deep learning models and exploring additional spectral analysis techniques. We also plan to validate our simulation results with experimental data from real soil samples.

\section{Acknowledgments}

We acknowledge the contributions of the USDA scientists for their guidance and support in this research. The spectral data generated in this study is available for further research and validation.

%% The Appendices part is started with the command \appendix;
%% appendix sections are then done as normal sections
%% \appendix

%% \section{}
%% \label{}

%% If you have bibdatabase file and want bibtex to generate the
%% bibitems, please use
%%
\bibliography{references}

%% Note: We're using BibTeX for references management, so manual bibliography is not needed.
\end{document}
